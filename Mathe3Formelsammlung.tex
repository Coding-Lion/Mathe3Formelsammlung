\documentclass[12pt, a4paper]{scrartcl}

\usepackage{mystyle, pgfplots, amssymb, todonotes, trfsigns}
\usetikzlibrary{angles, arrows}
\pgfplotsset{
  width=7cm,
  grid = major,
  axis lines = center,
}

% \setlength{\extrarowheight}{12pt}

\newenvironment{mathframed}{\empheq[box={\fbox}]{align*}}{\endempheq}
\renewcommand{\im}{\textcolor{myred}{i}}

\begin{document}

\newgeometry{margin=1.5cm}
\begin{titlepage}

  \includesvg[width=0.25\textwidth]{res/Grafiken/LogoHS_Esslingen}\\ \vspace{3cm}
  
  \begin{center}
    {\usekomafont{disposition}
      \Huge Formelsammlung Mathematik 3}
    \vspace{0.5cm}
    
    \begin{Large}
      Tim Hilt\\
      \vspace{0.4cm}
      \today\\
    \end{Large}
    
  \end{center}
\end{titlepage}
\restoregeometry

%%% Local Variables:
%%% mode: latex
%%% TeX-master: "Mathe3Formelsammlung"
%%% End:

\tableofcontents
\clearpage

\section{Grundlagen und Wiederholung}

\subsection{Allgemeine trigonometrische Umformungen; Additionstheoreme und Doppelwinkelformeln}

\bgroup
\def\arraystretch{1.5}
\begin{center}
  \begin{tabularx}{.7\textwidth}{rXl}
    \(\tan x\) & \(=\) & \(\frac{\sin x}{\cos x}\)\\
    \(\sin ^2 x + \cos ^2 x\) & \(=\) & \(1\)\\[1em]
    \(\sin ( x \pm y )\) & \(=\) & \(\sin x \cos y \pm \cos x \sin y\)\\
    \(\cos ( x \pm y )\) & \(=\) & \(\cos x \cos y \mp \sin x \sin y\)\\
    \(\tan ( x \pm y )\) & \(=\) & \(\frac{\tan x \pm \tan y}{1 \mp \tan x \tan y} = \frac{\sin ( x \pm y )}{\cos ( x \pm y )}\)\\[1em]
    \(\sin (2x)\) & \(=\) &  \(2 \sin x \cos x\)\\
    \(\cos (2x)\) & \(=\) & \(\cos ^2 x - \sin ^2 x = 1 - 2 \sin ^2 x = 2 \cos ^2 x - 1\)\\
    \(\tan (2x)\) & \(=\) & \(\frac{2 \tan x}{1 - \tan ^2 x}\)
  \end{tabularx}
\end{center}
\egroup

\subsection{Komplexe Zahlen}

\subsubsection{Darstellungsformen komplexer Zahlen}

\begin{center}
  \begin{tabularx}{\textwidth}{XX}
    \toprule
    Kartesische Form: & \(z = x + \im y\)\\
    Polarform; Polarkoordinaten: & \(z = r \cos \varphi + \im r \sin \varphi\)\\
    Exponentialform: & \(r e^{\im \varphi}\)\\
    \bottomrule
  \end{tabularx}

  \begin{tikzpicture}
    \begin{axis}[
      ymin = -0.5,
      xmin = -0.5,
      xmax = 2,
      ymax = 2,
      xtick distance = 1,
      ytick distance = 1,
      grid = none,
      xlabel = \(\operatorname{Re}\),
      ylabel = \(\operatorname{Im}\),
      extra x ticks = {1.6},
      extra y ticks = {1.3},
      extra x tick label = {\(x\)},
      extra y tick label = {\(y\)},
      width = \textwidth,
      ]
      \draw [->, >=stealth, very thick, myred] (axis cs:0,0) -- (axis cs:1.6, 1.3) node [sloped, above, midway] {\(r\)};
      \draw [dotted] (axis cs:0,1.3) -- (axis cs:1.6,1.3);
      \draw [dotted] (axis cs:1.6,0) -- (axis cs:1.6,1.3);
      \path (axis cs:1.6,1.3) node [myred, above] {\(z\)};
      \filldraw [fill = myred!20, draw = myred] (axis cs:0,0) -- (axis cs:0.5,0) arc [start angle=0, end angle=36.5, radius=2.7cm] -- cycle;
      \path (axis cs:0.3,0.11) node [myred] {\(\varphi\)};
    \end{axis}
  \end{tikzpicture}
\end{center}

\subsubsection{Umrechnung verschiedener Darstellungsformen ineinander}

\begin{mathframed}
  z &= r \cos \varphi + \im r \sin \varphi = r e^{\im \varphi}\\
  r &= |z| = \sqrt{x^2 + y^2}
\end{mathframed}

\begin{mathframed}
  \arg (z) = \varphi =
  \begin{cases}
    \arctan \left(\frac{y}{x}\right) & \text{für } x>0, y \text{ bel.}\\
    \arctan \left(\frac{y}{x}\right) + \pi & \text{für } x<0, y \geq 0\\
    \arctan \left(\frac{y}{x}\right) + \pi & \text{für } x<0, y<0\\
    \frac{\pi}{2} & \text{für } x = 0, y > 0\\
    - \frac{\pi}{2} & \text{für } x = 0,y < 0
  \end{cases}
\end{mathframed}

\subsubsection{Darstellung Sinus und Kosinus als komplexe Zahlen}

Zudem können Kosinus und Sinus auch dargestellt werden durch:

\begin{mathframed}
  \cos \varphi = \frac{e^{\im \varphi} + e^{- \im \varphi}}{2} \qquad
  \sin \varphi = \frac{e^{\im \varphi} - e^{- \im \varphi}}{2 \im}
\end{mathframed}

\subsection{Sinus Kardinalis}

\begin{minipage}{.5\linewidth}
Der Sinus Kardinals \(\operatorname{si}(x)\) ist definiert als

\[
  \operatorname{si}(x)=
  \begin{cases}
    \frac{\sin(x)}{x} & x \in \mathbb{R} \backslash 0\\
    1 & x = 0
  \end{cases}
\]
\end{minipage}%
\begin{minipage}{.5\linewidth}
  \missingfigure{Sinus Kardinalis, mit Python plotten}
\end{minipage}

\begin{minipage}{.5\linewidth}
  Eine spezielle Form ist die \(\operatorname{sinc}(x)\)-Funktion. Sie ist definiert als:
  \[
    \operatorname{sinc}(x)=
    \begin{cases}
      \frac{\sin(\pi x)}{\pi x} & x \in\mathbb{R} \backslash 0\\
      1 & x = 0
    \end{cases}
  \]
\end{minipage}%
\begin{minipage}{.5\linewidth}
  \missingfigure{\(\operatorname{sinc}(x)\)-Funktion, auch mit Python plotten}
\end{minipage}


\section{Verallgemeinerte Funktionen}

\subsection{Heaviside-Funktion}
\begin{minipage}{.5\textwidth}
  Die Heaviside-Funktion oder Einheitssprungfunktion ist definiert durch:
  \[
    \sigma (t) =
    \begin{cases}
      0 & t \in (-\infty, 0]\\
    1 & t \in (0, \infty)
  \end{cases}
\]
\end{minipage}\hfill%
\begin{minipage}{.5\textwidth}
  \centering
  \begin{tikzpicture}
    \begin{axis}[xmin = -5,
        xmax = 5,
        ymin = -1,
        ymax = 2,
        legend entries = {\(\sigma(t)\)},
        ]
      \addplot[domain=-5:0, myred, very thick] {0};
      \addplot[domain=0:5, myred, very thick] {1};
    \end{axis}
  \end{tikzpicture}
\end{minipage}

\begin{minipage}{.5\textwidth}
  Wird eine Funktion mit der Heaviside-Funktion multipliziert, so werden Teile der Funktion ausgeblendet.
\end{minipage}\hfill%
\begin{minipage}{.5\textwidth}
  \centering
  \begin{tikzpicture}
    \begin{axis}[]
      \addplot [domain=0:5, myred, very thick] {x^2+5};
      \addplot [domain=-5:0, blue, very thick] {0};
    \end{axis}
  \end{tikzpicture}
\end{minipage}

\begin{minipage}{.5\textwidth}
  Mithilfe der Heaviside-Funktion können Rechteckimpulse erstellt werden.

  \begin{framed}
    \[
      r(t) = \sigma (t-t_0) - \sigma (t-t_1)
    \]
  \end{framed}
\end{minipage}\hfill%
\begin{minipage}{.5\textwidth}
  \centering
  \begin{tikzpicture}
    \begin{axis}[
        ymax=1.5,
        xmin = -1,
        xmax = 3,
        xtick = {1,2},
        xticklabels={
          \(t_0\),
          \(t_1\),
        },
        ]
      \addplot [domain = -5:1, myred, very thick] {0};
      \addplot [domain = 1:2, myred, very thick] {1};
      \addplot [domain = 2:5, myred, very thick] {0};
      \draw [very thick, myred] (axis cs:1,0) -- (axis cs:1,1);
      \draw [very thick, myred] (axis cs:2,0) -- (axis cs:2,1);
    \end{axis}
  \end{tikzpicture}
\end{minipage}

\subsection{Dirac-Distribution}
\begin{minipage}{.5\textwidth}
  Die Dirac-Distribution ist definiert durch:
  \[
    \delta(x)=
    \begin{cases}
      0 & t \in \mathbb{R} \backslash 0\\
      1 & t = 0\\
    \end{cases}
  \]
\end{minipage}\hfill%
\begin{minipage}{.5\textwidth}
  \centering
  \begin{tikzpicture}
    \begin{axis}[
        xmin = -5,
        xmax = 5,
        ymin = -1,
        ymax = 2,
        legend entries = {\(\delta(t)\)},
        ]
      \addplot[myred, very thick] {0};
      \draw[->, very thick, myred, >=stealth] (axis cs:0,0) -- (axis cs:0,1);
    \end{axis}
  \end{tikzpicture}
\end{minipage}

Genauer wird die Dirac-Distribution durch eine Folge von Rechteckimpulsen hergeleitet, die den konstanten Flächeninhalt \(1\) besitzen, deren Breite dabei jedoch gegen \(0\) strebt, deren Höhe dafür aber gegen \(\infty\).

\begin{minipage}{.5\textwidth}
  Wird eine Funktion mit der Dirac-Distribution an einem Punkt \(t\) multipliziert, so wird die \textbf{gesamte Funktion, bis auf den Funktionswert an der Stelle \(t\) ausgeblendet!}
\end{minipage}\hfill%
\begin{minipage}{.5\textwidth}
  \centering
  \begin{tikzpicture}
    \begin{axis}[
        xmin = 0,
        xmax = 5,
        ymin = -1,
        ymax = 5,
        samples = 50,
        ]
      \addplot [gray] {(x-2)^2+2};
      \draw [->, >=stealth, myred, very thick] (axis cs:3,0) -- (axis cs:3,3);
    \end{axis}
  \end{tikzpicture}
  {\footnotesize \((t-2)^2+2 \cdot \delta(t-3)\)}
\end{minipage}

\subsection{Verallgemeinerte Ableitung}

Leitet man die Heaviside-Funktion ab entsteht die Dirac-Distribution.

\begin{framed}
  \[
    \dot{\sigma} (t) = \delta (t)
  \]
  \mybfred{Beim Ableiten ist insbesondere auf die innere Ableitung zu achten!}
\end{framed}

\subsubsection{Grafisches Ableiten verallgemeinerter Funktionen}

\begin{minipage}{.45\textwidth}
  Wird eine Funktion mit Unstetigkeitsstellen abgeleitet, so wird an der Sprungstelle ein Dirac-Impuls in Höhe und Richtung des Sprungs eingezeichnet. Dieser Impuls ist von der \(x\)-Achse aus zu zeichnen.
\end{minipage}\hfill%
\begin{minipage}{.45\textwidth}
  \begin{tikzpicture}
    \begin{axis}[
        axis lines = center,
        ymin = -1.5,
        ymax = 1.5,
        ytick distance = 0.5,
        xtick distance = 1,
        xlabel = \(t\),
        ylabel = \(y\),
        legend pos = outer north east,
        ]
      \addlegendentry{\(f(t)\)}
      \addlegendentry{\(\dot{f} (t)\)}
      \addplot [domain = -2:0, color = myred, very thick] {-0.5};
      \addplot [domain = -2:0, color= mygreen, very thick] {0};
      \addplot [domain = 0:1, myred, very thick] {x^2};
      \addplot [domain = 1:2, myred, very thick] {1};

      \addplot [domain = 0:1, color= mygreen, very thick] {2*x};
      \addplot [domain = 1:2, color= mygreen, very thick] {0};
      \draw [->, >=stealth, color= mygreen, very thick] (axis cs:0,0) -- (axis cs:0,0.5);
    \end{axis}
  \end{tikzpicture}
\end{minipage}

Jedoch kann die Dirac-Distribution mit unserem Wissensstand nicht weiter abgeleitet werden.

\subsection{Faltung}

"A convolution is an integral that expresses the amount of overlap of one function \(g\) as it is shifted over another function \(f\)." (\href{http://mathworld.wolfram.com/Convolution.html}{\texttt{http://mathworld.wolfram.com/Convolution.html}})

Die Faltung ist definiert durch:

\begin{mathframed}
  h(t) = f(t) \star g(t) = \int_{-\infty}^{\infty}f(\tau) \cdot g(t-\tau) d \tau
\end{mathframed}

Dadurch entsteht eine neue Funktion \(h(t)\). \(\tau\) ist eine Dummy-Variable! Beim Integrieren verschwindet sie und bildet die Funktion wieder auf~ \(t\) ab.\todo{Noch mehr schreiben!}

\subsection{Faltung mit der Dirac-Distribution}

\section{Fourier-Transformation}

Mithilfe der Fouriertransformation werden Funktionen aus dem Zeitbereich in den Frequenzbereich übersetzt:

\begin{mathframed}
  s(t)~\laplace~S(f) = \int_{-\infty}^{\infty} s(t) e^{- \im 2\pi ft} dt
\end{mathframed}

\subsection{Real- und Imaginärteil direkt berechnen}

In der Regel ist das Ergebnis einer Fouriertransformation eine komplexwertige Funktion. Natürlich lässt sich diese stets in Real- und Imaginärteil aufspalten --- die Anteile lassen sich aber auch direkt berechnen:

\begin{mathframed}
  \operatorname{Re} (z) &= \int_{- \infty}^{\infty} s(t)\ \cos (2\pi f t)\ dt\\[1em]
  \operatorname{Im} (z) &= \int_{- \infty}^{\infty} s(t)\ \sin (2\pi f t)\ dt
\end{mathframed}

\clearpage

\subsection{Fourier-Transformation von geraden- und ungeraden Funktionen}

Ist die zu transformierende Funktion \mybfred{gerade}, so ist die Transformierte \mybfred{rein reell und ebenfalls gerade}, aufgrund der Symmetrieeigenschaften des Kosinus. Zudem muss nicht mehr von \(- \infty\) bis \(\infty\) integriert werden. Es genügt das Integral von \(0\) bis \(\infty\) mit \(2\) zu multiplizieren. Ihre Berechnung reduziert sich dabei auf:

\begin{mathframed}
  S(f) = \operatorname{Re}(f) = 2 \int_{0}^{\infty} s(t)\ \cos (2\pi f t) dt
\end{mathframed}

Das selbe Prinzip lässt sich auf die Berechnung \mybfred{ungerader} Funktionen anwenden. Hier ist die Transformierte \mybfred{rein imaginär und ungerade}:

\begin{mathframed}
  S(f) = \im \operatorname{Im}(f) = -2\im \int_{0}^{\infty}s(t)\ \sin (2\pi f t) dt
\end{mathframed}

\clearpage
\subsection{Eigenschaften der Fourier-Transformation}

\bgroup
\def\arraystretch{1.5}
\begin{tabularx}{\textwidth}{XXX}
  \toprule
  \mybfred{Eigenschaft} & \mybfred{Zeitbereich} & \mybfred{Frequenzbereich}\\
  \midrule
  Linearität & \(C_1 s_1(t) + C_2 s_2(t)\) & \(C_1 S_1(f) + C_2 S_2(f)\)\\
  Zeitverschiebung & \(s(t-t_0)\) & \(e^{- \im 2\pi f t_0} S(f)\)\\
  Frequenzverschiebung & \(e^{\im 2 \pi f_0 t}s(t)\) & \(S(f-f_0)\)\\
  Amplitudenmodulation & \(s(t) \cos (2 \pi f_0 t)\) & \(\frac{1}{2}(S(f-f_0) + S(f+f_0))\)\\
  Ähnlichkeit & \(s(at)\) & \(\frac{1}{|a|}S \left(\frac{f}{a}\right)\)\\
  Zeitumkehr & \(s(-t)\) & \(S(-f)\)\\
  \midrule
  Differenziation in \(t\) & \makecell{
    \(\dot{s}(t)\)\\
    \(\ddot{s}(t)\)\\
    \(\vdots\)\\
    \(\frac{d^n}{dt^n}s(t)\)
  } & \makecell{
    \(\im 2\pi f S(f)\)\\
    \({(\im 2 \pi f)}^2 S(f)\)\\
    \(\vdots\)\\
    \({(\im 2\pi f)}^n S(f)\)
  }\\ \midrule
  Differenziation in \(f\) & \makecell{
    \((-\im 2 \pi t) s(t)\)\\
    \({(-\im 2 \pi t)}^2 s(t)\)\\
    \vdots\\
    \({(-\im 2 \pi t)}^n s(t)\)\\
  } & \makecell{
    \(\dot{S}(f)\)\\
    \(\ddot{S}(f)\)\\
    \(\vdots\)\\
    \(\frac{d^n}{dt^n}S(f)\)
  }\\ \midrule
  Multiplikation in \(t\) & \makecell{
    \(ts(t)\)\\
    \(t^2s(t)\)\\
    \vdots\\
    \(t^n s(t)\)\\
  } & \makecell{
    \(S'(f)\)\\
    \(\frac{S''(f)}{-\im 2 \pi}\)\\
    \vdots\\
    \(\frac{S^{(n)}}{{(-\im 2 \pi)}^n}\)
  }\\ \midrule
  Integration & \(\int_{-\infty}^t s(\tau) d \tau\) & \(\frac{1}{\im 2 \pi f} S(f) + \frac{1}{2} S(0) \delta (f)\)\\
  Faltung in \(t\) & \(s_1(t) \star s_2(t)\) & \(S_1(f) \cdot S_2(f)\)\\
  Faltung in \(f\) & \(s_1(t) \cdot s_2(t)\) & \(S_1(f) \star S_2(f)\)\\
  \bottomrule
\end{tabularx}
\egroup
\clearpage

\section{Laplacetransformation}

\section{z-Transformation}

\section{Statistik}

\subsection{Beschreibende Statistik}
\subsection{Wahrscheinlichkeitsrechnung}
\subsection{Schließende Statistik}

\end{document}

%%% Local Variables:
%%% mode: latex
%%% TeX-master: t
%%% End:
